\chapter{Game theory}
\section{Taxing Ships in 16th Century Denmark*}
\subsection*{Quick review}
So how do economists use game-theoretic models? Recall from the lecture that a \textit{game} consists of agents, actions, and payoff. We also know about sequential games, where actions earlier in the game impact the choices available later. 

Whenever we have a situation where one agent's choices depend on or affect the choices of another agent, we would apply ideas from game theory. When there is only one decision-maker, we would only use ideas from the previous chapters (unconstrained/constrained maximization, etc.). 

Suppose that I'm deciding how to spend my money at a store. If the store owner does not change its behavior based on what I choose to buy, we're back at Chapter X. However, if they adjust its prices based on my preferences over goods, then we would need concepts and ideas from game theory since my choices affect the store owner's choices. It may be helpful to think of game theory as a type of constrained maximization where other players' actions are a part of our constraints.

To me, the two key ideas in game theory are \textit{equilibrium} and \textit{backward induction}. An equilibrium state is one where no one could make themselves better off by playing some other strategy/action.\footnote{In class, we covered Nash equilibrium for pure and mixed strategies, as well as subgame perfect equilibriums in sequential games. There are also other types of equilibrium, such as trembling hand equilibrium, which is a subset of Nash equilibrium where every strategy must have a non-zero probability of being played (i.e., you don't always get to play the action you intended because your hands are ``trembling''). There is also Evolutionary Stable Strategies. Also a subset of nash equilibrium, a strategy is evolutionarily stable if, when it is played by most participants, it remains a better strategy compared to a mutant strategy played by a small proportion of participants \citep{Broom_2013_Gamerbio}} Backwards induction is when we take into account how others would respond to our moves, and then make moves based on that information.\footnote{You might ask, what if the agent is not sophisticated and cannot see far into the game tree to plan out their actions? If you're interested, I recommend checking out \citet{li_2017_OSP}.} We see the following two concepts in action in the following example.

\subsection*{Taxing Ships in 16th Century Denmark \citep{Haan_2012_Taxation}}

In the 16th century, foreign ships passing through \href{https://www.google.com/maps?q=%C3%98resund&=Search%20Google%20Maps}{Øresund}, the strait next to Copenhagen, had to make a stop at Helsingør (Yes, the same \href{https://en.wikipedia.org/wiki/Hamlet}{Elsinore} where \textit{Hamlet} takes place) and pay taxes to the Danish Crown based on the value of their cargo (usually 1-5\%). To encourage truth-telling, the Crown reserved the right to buy the entire cargo at the value the capitan reports. These are called The Sound Dues. 

Before I dive into the model in Hann et al.'s paper, feel free to \href{https://pbs.twimg.com/media/C1hNo_KUcAAJDQ9.jpg:large}{pause and ponder} for a second. How would you model this situation? What value does the capitan report? What is the capitan's equilibrium strategy? How much revenue does the Crown expect to collect?

Presuming that you've thought about this a bit yourself already, let's dive into \citet{Haan_2012_Taxation}. First, we see that this is a perfect scenario for game-theoretic modeling. We have two agents: the king and the capitan. For simplicity, I will refer to the king as he and the capitan as she. The capitan has one action: pick a value to report, and the king, upon learning this value, can either choose to tax the ship based on the declared value or buy the cargo. We presume that the king and the capitan are both trying to maximize the amount of money they get (here's where the maximization comes in). 

Suppose that the true value of the cargo is $v$, the reported value is $m$ (for message), the tax rate is $t$, and that the value of the cargo is the same for the capitan and the king.\footnote{The authors dive into the scenario where the cargo is worth more to the capitan than the king (because the king has to go through the trouble of seizing the ship and selling the cargo) in the paper. Remember, when you have a situation, it's always a good idea to work through a simple example before generalizing the model.} If the king buys the cargo, he receives $u_K(b) = v - m$; if he chooses to levy a tax, he receives $u_K(t) = mt$. Similarly, we have $u_C(b) = m - v$ and $u_C(t) = -mt$ for the capitan. Notice here that the capitan plays first, so we have the following game tree:

\begin{figure}[H]
    \caption{Game tree for \citet{Haan_2012_Taxation}}
    \centering
    \includegraphics[width = 3.5in]{taxgamertree.png}
\end{figure}

So now we've got a model of the sound dues; let's find the strategies that the king and the capitan will play in equilibrium. Why do we care about equilibrium strategy? Well, if we're out of equilibrium, then it means that either the king or the capitan can play some other strategy to improve their outcomes. We assume that the agents would have already played an alternative strategy if there is one that improves their payoffs. With that in mind, we introduce the following theorem:

\begin{theorem}
    The king is indifferent between playing buy or tax for any equilibrium message $m$.
\end{theorem}
I will first present an intuitive proof. Suppose that the king prefers to buy the cargo. Then, knowing this, the capitan should report a higher value $m$ so that she gets paid more. She should keep on increasing the value until the king is indifferent. Similarly, suppose that the king prefers to tax the cargo, then the capitan should report a lower $m$ so she has to pay taxes, and do this until that the king is indifferent. The mathematical proof follows:
\begin{proof}
    Suppose that this isn't true and that the king strictly prefers buying the cargo. We will attempt to show a contradiction that $v - m - mt \leq 0$. Suppose that $v - (1 + t)m > 0$, then there exist some $\epsilon$ such that $m' = m + \epsilon$ and $v - (1 + t)m' > 0$. Thus, the capitan can report $m'$ instead of $m$ and gain $m'-v > m - v$, which means that $m$ is not the optimal strategy for the capitan, and thus we are not in equilibrium. The proof for when the king prefers taxing the cargo is analogous and is left as an exercise for the reader. 
\end{proof}

Given that the king is indifferent, it follows that the expected payoff that the king gets whether he buys the cargo or not is the same. That is
\begin{align*}
    v - m &= mt \\
    m &= \frac{v}{1 + t}
\end{align*}

We know that as long as the capitan plays $m = \frac{v}{1 + t}$, the king would be indifferent. We now check what the king would need to do such that the capitan would want to play $m = \frac{v}{1+t}$. Given that the capitan has some probability $p(m)$ of playing ``buy", the expected utility of the capitan is
\begin{align*}
    EU &= p(m)(m - v) + (1 - p(m))(-tm)
\end{align*}
The capitan gets to choose $m$, so we differentiate the equation w.r.t. $m$ and find the optimum
\begin{align*}
    \pd{EU}{m} & = p'(m)(m-v) + p(m) + p'(m)mt -t(1-p(m)) = 0 \\
    & = p'(m)m - p'(m)v + p(m) + p'(m)mt -t + tp(m) = 0
\end{align*}
We know from above that the capitan needs to play $m(1+t) = v$ for the King to be indifferent, so we can substitute that in to get
\begin{align*}
    0 & = p'(m)m - p'(m)v + p(m) + p'(m)mt -t + tp(m) \\
    & =  p'(m)m - p'(m)m(1+t)  + p(m) + p'(m)mt -t + tp(m) \\
    & = p(m) -t + tp(m) = 0
\end{align*}
Rearranging gives us
\begin{align*}
    p(m)(1 + t) & = t\\
    p(m) & = \frac{t}{1+t}
\end{align*}
In conclusion, for any given tax rate $t$, the skipper will report $m = \frac{v}{1+t}$, and the king will buy the cargo with probability $\frac{t}{1+t}$ regardless of what was reported. Here, neither player has an incentive to deviate from their current strategy, and thus we have found the Nash equilibrium of the game. 

Since the capitan will under-report the true value of their cargo, the mechanism does not induce truth-telling. We see that regardless of what the king plays he is expected to receive $\frac{t}{(1+t)}m$. Thus if the king wants to implement some tax rate $t$, he needs to find tax rate $t^*$ such that $\frac{t^*}{1+t^*} = t$.

More details on this problem can be found in \citet{Haan_2012_Taxation}, but hopefully, this small paragraph gave you a taste of how game-theoretic modeling can be applied to real life situations. 
