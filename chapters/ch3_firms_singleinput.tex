\chapter{Firms with a Single Input}
We start with one of the simplest economic models: a firm in a perfectly competitive market with a single input. Firms are one of the most basic components of the economy. Firms purchase inputs, convert those inputs into outputs via a production function, and then sell those goods to make a profit. This also gives firms a clear objective function: profit. To make things simple, we assume that the firm only requires one input to produce their good, so the only choice that the firm makes is how much input to purchase. 

\section{Model Setup} \label{sec:single_firm_setup}
We will formalize a mathematical model of the firm. We assume that the firm can choose to hire $L$ units of labor at a wage $w$, which is the price per unit of labor. The firm has a production function, $f(L)$, which takes the units of labor as an input, and returns some unit of product. We assume that the production function is continuous and twice differentiable, increasing, $f'(L) > 0$, and concave $f''(L) < 0$, for any $L$. The firm can then sell each unit of product at a price $p$. The firm takes both $p$ and $w$ as exogenous variables. We can then define the firm's profit, $\pi$, as follows:
\begin{align}
    \pi(L; p, w) = p f(L) - w L \label{eq:profit_def}
\end{align}
Notice that our profit function has one endogenous input and two exogenous inputs. $L$, the units of labor hired, is the choice variable for the firm. $p$, the price of the product, and $w$, the wage cost of labor, are both exogenous variables. While we write them explicitly here, we will often only write $\pi$ and the arguments to it are implicit. 

While this seems like a fairly simple model, there are some pretty important assumptions underlying it.

\subsection*{Assumptions}
\begin{description}
    \item[Perfectly competitive market for output] Notice that the firm treats the price $p$ as exogenous. That is, no matter how much the firm produces, they can always sell goods at price $p$, and \emph{only} at price $p$. This means, first, that the firm is a \vocab{price-taker}, which means that they cannot set a price $p$ that differs from the market price $p$. The underlying assumption here is that there are enough other firms that if this firm were to raise its price, all of the customers would buy from other firms and our firm would sell 0. This assumption also entails that the amount our firm produces does not affect the market price, which can be taken to mean that there are many other firms producing a lot of the same good, so our firm's decisions do not have a noticeable effect on $p$. 
    \item[Perfectly competitive labor market] Similar to the above market for goods, we also assume that the market for labor is perfectly competitive. That is, the firm can only hire at the wage $w$, and that no matter how much labor the firm hires, the wage will not change. 
    \item[No liquidity constraints] We assume that the firm has the ability to hire as much labor as they want, and all that matters is the final profit. That is, the firm does not have some fixed budget for labor at the beginning. This can be thought of as a firm's ability to borrow at zero interest to finance labor so long as the loan is paid back. This assumption is key to the firm problem, as it allows us to deal with an unconstrained maximization problem rather than requiring a budget constraint for the firm. 
    \item[Diminishing marginal returns to consumption] This was expressed mathematically as $f''(L) < 0$. In real terms, this says that each additional unit of labor contributes less additional production than the previous unit of labor did, and represents a sort of ``too many cooks in the kitchen'' effect. Notice however that we will assume $f'(L) > 0$, so even if each additional unit of labor contributes less additional output than the previous unit, adding more units of labor can never make us produce less output. 
\end{description}

One reasonable question to ask with all of these assumptions in place is whether they are realistic assumptions. The answer is that they probably are not all perfectly realistic. However, there are cases where these assumptions might be close enough. Consider the market for corn, for example. Each individual farmer's corn production has a negligible effect on the market as a whole, and they have enough money every year to grow as much corn as is profitable. However, we will see that even if these assumptions are not all realistic, they help simplify the model so that we can solve it and gain some useful insights about the mechanics of this economy. 

\section{Solving the Model} \label{sec:solving_model}
Now that we have setup the model and established the underlying assumptions, we can begin to solve the model. Our objective function is $\pi$, and our choice variable is $L$, so we can write our maximization problem as follows:
\begin{align}
    \max_{L} \pi(L; p, w) = \max_{L} p f(L) - w L
\end{align}

\subsection*{First order conditions}
Our first order condition is that the derivative of the objective function with respect to our choice variable is 0, which is
\begin{align}
    \partials{\pi}{L} = 0 \label{eq:profit_foc}
\end{align}
We can plug in our function for $\pi$ using our definition in \ref{eq:profit_def} and differentiate with respect to $L$ to obtain,
\begin{align}
    \partials{\pi}{L} = p f'(L) - w = 0
\end{align}
This is our first order condition for an optimal profit. With some rearranging, we obtain that
\begin{align}
    p f'(L) = w 
\end{align}
Notice that this yields a very useful interpretation. The left hand side is the marginal revenue from an additional unit of labor. That is, for a small amount more labor, we produce $f'(L)$ more goods, which are sold at a price $p$. On the right hand side is the marginal cost for an additional unit of labor, because for a small amount more labor, we pay that labor a wage $w$. What this tells us is that when a firm is profit maximizing, \textbf{marginal revenue equals marginal cost}. 

Notice further that this implicitly defines our optimal choice of labor, which we will denote $L^*$. Because we do not know the functional form of $f$, we cannot yet write an explicit definition of $L^*$, but it is implicitly defined by the first order condition,
\begin{align}
    p f'(L^*) = w \label{eq:general_foc}
\end{align}
Observe further that the value of $L^*$ depends on the values of $p$ and $w$, so we can express it as a function of the exogenous variables, $L^*(p, w)$. 

\subsection*{Second order conditions}
In order to verify that $L^*$ is indeed a global maximum, and not a local maximum or even a minimum, we need to verify the second order conditions. Namely, that
\begin{align*}
    \partials{^2\pi}{L^2} < 0
\end{align*}
To do so, we can just differentiate the first derivative from \ref{eq:profit_foc} with respect to $L$ again to obtain, 
\begin{align}
    \partials{^2\pi}{L^2} = pf''(L) \label{eq:generalized_soc}
\end{align}
We need the above derivative to hold with respect to any $L$. However, we assumed that $f''(L) < 0$, and since $p > 0$, then we have that $pf''(L)< 0$, which satisfies our second order conditions.

This tells us that $L^*$ defines a global maximum, which means that the firm is indeed profit maximizing! 

\subsection*{Some intuition}
While the above provides the mathematical technique for how to maximize, it is useful to get some intuition about what is actually happening here. To do so, we can think about the firm's decision process.

Suppose that you are running the firm but do not know any calculus. However, you do know how much you will produce for a given amount of labor and how much money you will make. One way you might decide how much labor to hire is to think what will happen if you hire one additional worker. At each point, you ask yourself whether you will make more money by hiring an additional worker or less money by hiring an additional worker. What determines this? Whether the contribution of that additional worker is greater than the additional cost of hiring that worker. Mathematically, you would hire an additional worker so long as, $p f'(L) > w$. 

However, you also know that because each additional worker's contribution is less than the last worker hired, at some point you will stop hiring workers. This is the second order condition at work. So, if the increments are small enough, you will eventually reach a point where hiring an additional worker makes no difference. That is, that $p f'(L) = w$, which is the first order condition. At this point, you should stop hiring workers because any more workers that you hire will cost more than they produce. 

At some level, all economic optimization methods boil down to this process. Thinking about what will happen if increase some quantity by an infinitesimal amount, and requiring that doing so makes no difference. 

\section{Specific production functions}
In the above section we showed how to set up and solve for a general production function, $f$. However, because of this generality, we were unable to obtain an explicit formulation of $L^*$. To do so, we need to specify the functional form of $f$, and we offer some examples here of specific functional forms that $f$ could take.

\subsection*{Single-variable Cobb-Douglas}
One of the most common production functions used in economics is the Cobb-Douglas production function. Although typically the production function is in multiple variables, we present a single variable version of that production function here. 
\begin{align}
    f(L) = L^\alpha, \, 0 < \alpha < 1 
\end{align}
In this expression, we can interpret $\alpha$ as a constant measuring the productivity of labor. We will see that $0 < \alpha < 1$ is necessary to ensure that the first and second order conditions hold. We can write the profit function,
\begin{align}
    \pi(L) = p L^\alpha - w L
\end{align}
We will now solve for the optimum.
\begin{description}
    \item[First order condition] The first order condition for a maximum is
    \begin{align}
        \partials{\pi}{L} = p \alpha L^{\alpha - 1} - w = 0
    \end{align} 
    Rearranging to solve yields
    \begin{align}
        L^* = \left(\frac{w}{p \alpha}\right)^{\frac{1}{\alpha - 1}} = \left(\frac{p \alpha}{w}\right)^{\frac{1}{1 - \alpha}} \label{eq:cobb_douglas_optimized}
    \end{align}
    Where the second equality holds by taking the reciprocal of the inside term and negating the exponent. 
    \item[Second order conditions] We should verify that $L^*$ is indeed a maximum by checking the second order conditions. We can take the second derivative of $\pi$ with respect to $L$ to obtain,
    \begin{align}
        \partials{^2 \pi}{L^2} = p \alpha (\alpha - 1) L^{\alpha - 2}
    \end{align}
    Now we can determine the sign of the above expression by examining each of the terms. $p > 0$ and $\alpha > 0$ by assumption. $L^{\alpha - 2} > 0$ if we assume that we hire at least some labor. And $\alpha - 1 < 0$ by assumption that $\alpha < 1$. So, we multiply three positive terms and a negative term, which means that the entire expression is negative. This satisfies our second order conditions. 
\end{description}

\subsection*{Log production}
Another relatively simple production function is the log production function, which takes the form,
\begin{align}
    f(L) = \log(L)
\end{align}
We will solve this optimization problem in a slightly different way than we did for the Cobb-Douglas case. Because we have already solved for the general first and second order conditions in \ref{sec:solving_model}, we can just plug in our derivations.
\begin{description}
    \item[First order conditions] The only expression we need to plug into the generalized first order condition is $f'(L)$, which is,
    \begin{align*}
        f'(L) = \frac{d \log(L)}{dL} = \frac{1}{L}
    \end{align*} 
    Plugging this into \ref{eq:general_foc} yields,
    \begin{align*}
        \frac{p}{L^*} = w \iff L^* = \frac{p}{w}
    \end{align*}
    \item[Second order conditions] To test the second order conditions, all we need is to solve for $f''(L)$, which is,
    \begin{align*}
        f''(L) = \frac{d^2 \log(L)}{d L^2} = - \frac{1}{L^2}
    \end{align*}  
    Then plugging into the generalized second order conditions, \ref{eq:generalized_soc}, yields
    \begin{align}
        \partials{^2 \pi}{L^2} = - \frac{p}{L^2}
    \end{align}
    Since $L^2 > 0, p > 0$, this must be negative, which satisfies the second order conditions. 
\end{description}

\section{Comparative statics}
Now that we have solved for the optimal quantity of labor used for a given price and wage, we can see how that optimal quantity changes for a given change in exogenous variables. That is, we will take a \vocab{comparative static}. We will first do so in the specific case of a Cobb-Douglas production function, and then we will see how we can do so for a general production function. 

\subsection*{Cobb-Douglas}
Because we can solve for the optimal quantity of labor explicitly as a function of price and wage, we can consider what happens with a small change in either. Before we start, it will be useful for us to rewrite \ref{eq:cobb_douglas_optimized} by writing each of the terms as a product of individual exponents,
\begin{align}
    L^* = \left(\frac{p \alpha}{w}\right)^{\frac{1}{1 - \alpha}} = \left(p^{\frac{1}{1 - \alpha}}\right) \left(\alpha^{\frac{1}{1 - \alpha}}\right) \left(w^{\frac{1}{\alpha - 1}}\right) \label{eq:cobb_douglas_optimized_simple}
\end{align}
This will make it easier for us to take derivatives with respect to each of the variables. 
\begin{description}
    \item[Price] To determine the effect of a small change in the price of the good, we can take the derivative of $L^*$ with respect to $p$. This yields,
    \begin{align*}
        \frac{dL^*}{dp} = \left(\frac{p^{\frac{\alpha}{1 - \alpha}}}{1 - \alpha}\right)\left(\alpha^{\frac{1}{1 - \alpha}}\right) \left(w^{\frac{1}{\alpha - 1}}\right)
    \end{align*}
    You can check that each of the terms above is positive to obtain that the entire expression is positive. So, all else equal, an increase in price will lead to the firm using more labor. This makes sense because with a higher price, the marginal revenue for the good increases, which means that the firm should hire more labor to produce more output. 
    \item[Wage] We can follow a similar procedure as above, differentiating $L^*$ with respect to the wage, $w$,
    \begin{align*}
        \frac{dL^*}{dw} = \left(\frac{p \alpha}{w}\right)^{\frac{1}{1 - \alpha}} = \left(p^{\frac{1}{1 - \alpha}}\right) \left(\alpha^{\frac{1}{1 - \alpha}}\right) \left(\frac{w^{\frac{2 - \alpha}{\alpha - 1}}}{\alpha - 1}\right)
    \end{align*}
    Notice that once again, every term is positive with the exception of $\alpha - 1$, which is negative. So, multiplying a series of positive terms with a single negative term means that the entire expression is negative.
\end{description}

\subsection*{Implicit Function Theorem}
While the above example tells us how price and wages affect the quantity of labor demanded in the specific case of the Cobb-Douglas production function, it seems that the specific form of the production function should not matter. After all, it would be very strange if there were a production function where prices increased and the quantity of labor decreased. But how can we determine this change if we do not know the form of $f$?

To gain some intuition for how we might approach this problem, let's look back at the first order condition,
\begin{align*}
    p f'(L^*) = w
\end{align*}
Now consider what happens if we increase $w$ by a little bit. The right hand side has increased, so the left hand side must also increase by a small amount to ensure the equality holds. However, $p$ is fixed, so it must be that the value of $L^*$ has changed. This makes sense because recall that with the optimized choice of $L$, $L^*$ is an implicit function of $p$ and $w$, so the value of $L^*$ changes.  In particular, it must have decreased, because we know that $f'' < 0$, which means that $f'$ decreases as $L$ increases, so to make $f'$ smaller, the value of $L^*$ must decrease. Notice that we were able determine how $L^*$ would change without having to know anything about the actual value of $L^*$ or the form of $f$. This is the idea behind using implicit differentiation to calculate the change. 

While this is an informal argument, we can formalize it with the \vocab{implicit function theorem}. While we do not offer a formal state of the theorem, it essentially states that if we have an equality of the form $f(x) = c$ for some constant $c$, and $f$ is sufficiently ``well-behaved'' (do not worry about what this means formally, and assume that all functions we deal with satisfy the necessary conditions), then we can express the value of $x$ as a function of $c$. This also means that we can use implicit differentiation to solve for the relatioship between the two variables.

In the context of our problem, it tells us that by implicity differentiating both sides of the first order condition, we can determine how $L^*$ changes with respect to wage and price, without even knowing the actual value of $L^*$!

\begin{description}
    \item[Wage] We will first formalize the above example of the wage change. To do so, we implicitly differentiate both sides of the FOC with respect to $w$. 
    \begin{align*}
            \frac{d}{dw} \left(p f'(L^*(w))\right) = \frac{d}{dw} w
    \end{align*}
    We can start by doing both sides separately. The derivative on the right hand side is simple, 
    \begin{align*}
        \frac{d}{dw}(w) = 1
    \end{align*} 
    The left hand side requires us to use the chain rule. We explicitly write $L^*$ as a function of $w$ in this case to make things clear.
    \begin{align*}
        \frac{d}{dw} \left(p f'(L^*)\right) = p f''(L^*) \frac{dL^*}{dw}
    \end{align*}
    Because the FOC must hold with equality, a small change in $w$ must lead to both sides staying the same. So, we can set the above expressions equal to each other,
    \begin{align*}
        p f''(L^*) \frac{dL^*}{dw} = 1
    \end{align*}
    Now, we can solve for $\frac{dL^*}{dw}$. Rearranging yields,
    \begin{align*}
        \frac{dL^*}{dw} = \frac{1}{p f''(L^*)}
    \end{align*}
    We can also determine the sign of $\frac{dL^*}{dw}$. Observe that $f''(L^*) < 0$ by assumption on our production function, and $p > 0$, so we have that overall the expression must be negative. That is,
    \begin{align*}
        \frac{dL^*}{dw} < 0
    \end{align*}
    This tells us that holding everything else constant, a small increase in the wage decreases the amount of labor that the firm demands. 

    \item[Price] We can perform a similar calculation with the price of the output good. Differentiate both sides of the FOC with respect to $p$,
    \begin{align*}
        \frac{d}{dp} \left(p f'(L^*)\right) &= \frac{d}{dp} w \\
        \implies p f''(L^*) \frac{d L^*}{dp} + f'(L^*) &= 0
    \end{align*} 
    Since $w$ is an exogenous variable, it does not depend on $p$ so the right hand side is clearly 0. The left hand side has a product, so we use the product role to implicitly differentiate. So we can solve for, $\frac{dL^*}{dp}$,
    \begin{align*}
        \frac{dL^*}{dp} = -\frac{f'(L^*)}{p f''(L^*)}
    \end{align*}
    Before reading further, try figuring out the sign of $\frac{dL^*}{dp}$ yourself. It should be a fairly straightforward exercise in examining the assumptions we made about $f$ and your result should match with your intuitions.

    By assumption, we have that $f'(L^*) > 0, p > 0$ and we know that $f''(L^*) < 0$, so the entire term must be positive. To check that this accords with our intuitions, if the price of the good increases, the firm should produce more of that good because the marginal revenue has increased. 
\end{description}

If you observe both of our comparative statics above, you may notice some similarities between them. For instance, the denominator in both is the same. It turns out that we can say generally how to take a comparative static for a given objective function. We show this below.

\begin{proposition*} \label{prop:single_var_static}
    Given a twice-differentiable increasing and concave objective function $F(x; z)$, with choice variable $x$ and exogenous variable $z$, and $x^*(z) = \argmax_{x} F(x; z)$, then $\frac{dx^*}{dz} = - \frac{F_{xz}}{F_{xx}}(x^*(z); z)$. 
\end{proposition*}

The above yields a general way of finding the comparative static of an optimized choice variable $x^*$ with respect to an exogenous variable $z$ for any given single variable optimization problem. We will prove the above result.

\begin{proof}
    This proof follows straightforwardly from taking first order conditions and then using the multivariate chain rule to implicitly differentiate. 

    The first order condition for an optimal $x$ is
    \begin{align*}
        F_x(x^*, z) = 0
    \end{align*}
    Now, we totally differentiate both sides with respect to $z$. The right hand side is clearly 0. The left hand side uses the multivariate chain rule.
    \begin{align*}
        F_{xx}(x^*, z) \frac{dx^*}{dz} + F_{xz}(x^*, z) = 0
    \end{align*}
    Rearranging for $\frac{dx^*}{dz}$ completes the proof,
    \begin{align*}
        \frac{dx^*}{dz} = -\frac{F_{xz}(x^*, z)}{F_{xx}(x^*, z)}
    \end{align*}
\end{proof}

One useful component to notice is that because we have assumed the second order conditions to hold, $F_{xx} < 0$, then the sign of the comparative static is the same as the sign of $F_{xz}$. This is a fairly powerful result that tells us how to very straightforwardly find the comparative statics for a single variable optimization problem. One important thing to note however, is that above, $F$ would be akin to profit, $\pi$, in the firm's problem because it is the objective function. This does \emph{not} hold for the production function $f$. 

\section{Value function}
In all of the above, we find that we can write the optimal amount of labor demanded as a function of the exogenous variables. However, we could do the same for all of the other functions. For example, we could define the optimal production function in terms of the exogenous variables as well,
\begin{align*}
    f^*(w, p) = f(L^*(w, p))
\end{align*}
And we can do the same for the profit function to determine how much profit the firm makes in terms of the exogenous parameters,
\begin{align*}
    \pi^*(w, p) = p f(L^*(w, p)) - w L^*(w, p)
\end{align*}
Since these are all functions of the exogenous variables, we can take comparative statics of each of them by straightforward differentiation and the chain rule. For example, to see how the quantity produced changes for an increase in price, we have
\begin{align*}
    \frac{df^*}{dp} = \frac{d f(L^*)}{dp} = f'(L^*) \frac{dL^*}{dp} = -\frac{f'(L^*)^2}{p f''(L^*)}
\end{align*}
Where the last equality holds by plugging in the value of $\frac{dL^*}{dp}$ that we obtained previously. 

Now we can do the same for the optimized profit function. This optimized version of the profit function, which we can denote as $\pi^*(w, p)$, is referred to as the value function. In general, the \vocab{value function} refers to the value of the objective function when making the optimal choice. The value function is always a function of the exogenous variables. 

To find the comparative statics for the value function, in this case profit, we can use a result known as the \vocab{envelope theorem}. Informally, the envelope theorem says that when calculating how the value function is affected by a change in an exogenous variable, we only need to look at the direct effect of the exogenous variable, and can treat our choices as fixed. We can write this more formally for the case of a single variable (although note that it generalizes to multiple variables),

\begin{theorem*}[Envelope] \label{thm:envelope}
    Let $F(x; z)$ be an objective function with choice variable $x$ and exogenous variable $z$. Let $V(z) = \max_x F(x; z)$ be the value function and $x^* = \argmax_x F(x; z)$ be the optimizing choice of $x$. Then,
    \begin{align*}
        \frac{dV}{dz} = \partials{F}{z}(x^*, z)
    \end{align*}
\end{theorem*}
Notice that in the last line, we are taking the partial derivative of $F$ with respect to the argument $x$, and then evaluating that derivative at the point $(x^*, z)$. 

We will apply this to examine how profit changes with respect to price. Let $\pi^*$ be the value function. First, we compute the partial derivative of profit $\pi$ with respect to the price,
\begin{align*}
    \frac{d\pi(L; p, w)}{dp} = \frac{d(p f(L) - w L)}{dp} = f(L)
\end{align*}
The envelope theorem tells us that the comparative static on the value function is equal to the above evaluated at $L^*$, 
\begin{align*}
    \frac{d \pi^*}{dp} = f(L^*)
\end{align*}
In other words, if prices increase by a small amount, the additional profit that we earn will be equal to how much we are producing. 

We will now prove the envelope theorem in the single variable case.

\begin{proof}
    First, observe that we can write $V(z) = F(x^*(z), z)$ because $x^*$ is the maximizing choice. We can use the multivariate chain rule to totally differentiate $V$ with respect to $z$
    \begin{align*}
        \frac{dV}{dz} = \frac{\partial F}{\partial x}(x^*(z), z) \frac{d x^*}{dz} + \frac{\partial F}{\partial z}(x^*(z), z)
    \end{align*}
    However, note that the first order condition for optimization is
    \begin{align*}
        \frac{\partial F}{\partial x}(x^*(z), z) = 0
    \end{align*}
    So, plugging into above yields,
    \begin{align*}
        \frac{dV}{dz} = 0 \frac{d x^*}{dz} + \frac{\partial F}{\partial z}(x^*(z), z) = \frac{\partial F}{\partial z}(x^*(z), z)
    \end{align*}
\end{proof}

Essentially the envelope theorem says that because we are at an optimum, a small change in our choice variable does not change the value of the overall objective function because of the first order condition. In the context of profit, this tells us that the effect of a change in price is purely the increase in price from the units that we are already producing, $f(L^*)$. Note that this does \emph{not} say that our choice of labor or that our production amount is not changing. Rather, it says that becuase we are at an optimum, the derivative of the profit with respect to $L$ must be 0. This means that a small change in $L$ will not affect the overall profit, so the only effect is the mechanical effect of increasing the price of the current units sold.

\section{Policy applications}
Now that we have a basic model of the firm setup, we can augment it slightly to add some common policy questions and see how they affect the firm's output. In particular, we can examine ƒixed costs and taxes on firms. For each addition to the basic model, we can ask whether it changes the firm's decisions or its profits. 

\subsection*{Fixed costs}
Suppose that firms have some fixed cost, $b$, that must be paid. This might be the cost of purchasing land, a factory, or filling out paperwork. First, we will assume that the firm must operate. In this case, the profit function is given by,
\begin{align*}
    \pi = p f(L) - w L - b
\end{align*}
Notice however, that this is a monotonic transformation of the profit function, which means that the optimal choice remains the same. We can verify this by taking first order conditions,
\begin{align*}
    \partials{\pi}{L} = pf'(L^*) - w = 0 \implies pf'(L^*) = w
\end{align*}
This is the same as the first order conditions in the basic firm model, which tells us that the optimal choices are the same. In particular, this means that profits are simply $b$ dollars lower than in the standard case.

However, so far we have assumed that the firm is forced to operate. In reality, if profits are negative, the firm would likely shut down. So, if the choice of $L^*$ implicitly defined above yields a positive profit, then the firm will choose $L^*$. However, if it yields a negative profit, then the firm will shut down and hire 0 labor. So, the choice of labor is instead given by,

\begin{align*}
    \begin{cases}
        L^* & \text{ if } p f(L^*) - w L^* - b \geq 0 \\
        0 & \text{ o/w }
    \end{cases}
\end{align*}

\subsection*{Tax per unit sold}
In both the United States and Europe, taxes are levied on many goods at the point of sale. In the United States, this takes the form of a sales tax, while the European Union uses a Value-Added Tax. Here, we model the effect of a tax on sales on firm production. 

Suppose that the government levies a tax of $\tau$ on each unit of production sold. If the market price is still $p$, then this means that the firm really receives $p - \tau$ dollars per unit sold. Notice that we could also use this to model a per-unit subsidy, which would simply be the case where $\tau < 0$. We assume that the market price does not adjust to the imposition of this tax, and that $\tau < p$.  This means that the firm's profit function is now,
\begin{align*}
    \pi = (p - \tau) f(L) - w L
\end{align*}
We can once again take first order conditions to obtain,
\begin{align*}
    \partials{\pi}{L} = (p - \tau) f'(L^*) - w = 0 \implies (p - \tau) f'(L^*) = w
\end{align*}
Notice that this is almost exactly the same as the original first order conditions, except that we've replaced $p$ with $p - \tau$. This is because the firm is facing essentially the same problem, except instead of facing a price $p$ for each good, they face $\tilde{p} = p - \tau$ as the price. 

One question we might have is whether this leads to higher or lower labor consumption that in the normal case. If we had an explicit functional form, we could directly compare for the resulting values of $L^*$. However, this is more difficult without an explicit functional form. A nice trick though is take comparative statics. Notice that the original case with no tax is just the special case of $\tau = 0$. So, if we take comparative statics with respect to $\tau$, we could determine the direction of the change in labor usage. To do so, we implicitly differentiate the first order condition with respect to $\tau$,
\begin{align*}
    &\frac{d}{d\tau}\left((p - \tau) f'(L^*)\right) = \frac{d}{d\tau} w \\
    \implies& (p - \tau) f''(L^*) \frac{dL^*}{d\tau}- f'(L^*) = 0 \\
    \implies& \frac{dL^*}{d\tau} = \frac{f'(L^*)}{(p - \tau) f''(L^*)} 
\end{align*}
There are a few things to notice. The first is that since we assumed $\tau < p$, then $\frac{dL^*}{d\tau}$ is negative. That is, the higher the tax, the less labor the firm hires. Since the production function is monotonically increasing in $L$, then this also tells us that as taxes are higher, the firm produces less output.

The second thing to notice is that if we were to differentiate with respect to the market price, $p$, instead, we would get,
\begin{align*}
    \frac{dL^*}{dp} = - \frac{f'(L^*)}{(p - \tau) f''(L^*)} = -\frac{dL^*}{d\tau}
\end{align*}
That is, the effect of a change in the tax is exactly the negative of the effect of a change in price. This is because the firm only cares about the after-tax price, in which case a tax increase is essentially the same as a price decrease.

Finally, a key point to notice is that if you were to compare $\frac{dL^*}{d\tau}$ here to the expression for $\frac{dL^*}{dp}$ in the model without taxes, you would get the same negative relationship except with $p$ replaced with $p - \tau$. This is to emphasize the point that if you know how the firm would respond to $p$, you can essentially replace $p$ with $p - \tau$ everywhere if you impose a tax. 

\subsection*{Tax on labor}
Rather than a tax on the end product, we can instead consider what would happen with a tax (or subsidy) on labor. This is modeled in essentially the same way as the tax on production, except now we impose a tax $\tau$ on each unit of labor, so that the total cost of one unit of labor is given by $w + \tau$. To avoid degenerate cases, we assume that $w + \tau > 0$ (this could be negative if we instead subsidized labor with $\tau < 0$). Then our profit function is given by
\begin{align*}
    \pi = p f(L^*) - (w + \tau) L
\end{align*}
Much like the case of a tax on output however, all we have done is change the effective price of labor. This means that rather than re-calculating everything, we can just plug in $w + \tau$ in place of $w$ everywhere in our original model. This immediately yields the first order condition as,
\begin{align*}
    p f'(L^*) = w + \tau
\end{align*}
Furthermore, notice that $\tau$ is essentially the same as $w$ from a mathematical perspective. That is, we could swap the symbols and the math would remain the same. This means that the comparative statics are the same, so we can just use our original expression for $\frac{dL^*}{dw}$ (which does not actually depend on $w$),
\begin{align*}
    \frac{dL^*}{d\tau}= \frac{1}{pf''(L^*)}
\end{align*}
These calculations are useful shortcuts, and can be replicated by our normal approach. 

\subsection*{Tax on profits}
Finally, we will consider a proportional tax on profits. That is, we impose a tax, $\tau$, on each dollar of profits earned. We assume that $\tau < 1$. This means that the firm's after tax profits are given by,
\begin{align*}
    \pi = (1 - \tau) \left(p f(L) - w L \right)
\end{align*}
However, notice that if $\tau < 1$, this is just a monotonic transformation of the original profit function, so the optimal choice of $L$ remains the same. This tells us that \emph{firm behavior is unchanged by a tax on profits}. However, you may have seen that economists are often opposed to high corporate income taxes. Why would this be the case if we can impose large taxes with no distortionary effects? This is because our model only has one period, and ignores important factors like investment in future productivity, and the risk of starting a firm in the first place.

\section*{Recap}
We can now calculate how profit-maximizing firms will choose how many workers to hire, and we have also examined how the firm's labor demand, production, and profit will change in response to variations in price and wages. We have observed these changes not only with specific production functions, but also obtained more general results for how firms should respond to changes. These are already some very powerful economic modeling tools that can formalize existing economic intuitions and suggest how we might capture more complex results. In the next chapter, we will develop the model slightly by considering firms who not only choose labor, but multiple inputs. 
